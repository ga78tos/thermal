% !TeX root = ../main.tex
% Add the above to each chapter to make compiling the PDF easier in some editors.

\chapter{Future Work}\label{chapter:future_work}
The experiments performed over the course of this thesis have returned good evidence on the performance of the GMPT and PMPT algorithm. Especially the notebook has provided reliable data which constitutes a good basis upon which the performance of the GMPT algorithm was evaluated. The data from the Raspberry Pi was not as clear.\\
\hspace*{0.5ex}\hspace{0.5ex} A first option for future work is the improvement of the framework, specifically its adaption to the Raspberry Pi platform. Currently, experiments on this platform are associated with a major overhead due to the slow frequency switch. This leads to a significant deterioration of the temperature data and makes its clear analysis difficult. As of today, the Raspberry Pi 3B does not provide direct frequency scaling options via the CPUFreq interface (except for setting minimum or maximum frequency) and therefore the framework resorts to the vcmailbox driver functionality, which in its current implementation is the source of the lengthy frequency switch. If the Raspberry Pi is to be used in future experiments for similar purposes the communication with this driver should be looked into, in order to determine potential steps to optimize the interaction with it. Alternatively, future generations of the Raspberry Pi may provide a more flexible CPUFreq driver. In this case, the frequency changes could be implemented in the same fashion as they were on the notebook (likely resulting in similarly good performance).\\
\hspace*{0.5ex}\hspace{0.5ex} Ultimately I would propose a different option to work on in the future. The framework may provide an easy way to test these schedules on multiple platforms but it only simulates a real-time system. I don't believe more experiments on the Raspberry Pi will really add a lot of value to the evaluation of the GMPT algorithm and its schedules as Raspberry Pi's themselves are no closer to embedded hard real-time systems than any notebook. To all intents and purposes a Raspberry Pi is just a desktop PC without a fan and without a case. An alternative to this would be to test the schedules on a true real-time systems. Boards, similar to the Raspberry Pi, are available, which run on an actual RTOS. Using such a platform for future experiments requires more setup work but eliminates any influences the OS or the framework have on timing and temperature. These experiments would give unequivocal proof for the performance of the GMPT schedules.
\afterpage{\null\newpage}