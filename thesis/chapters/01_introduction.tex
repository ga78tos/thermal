% !TeX root = ../main.tex
% Add the above to each chapter to make compiling the PDF easier in some editors.

\chapter{Introduction}\label{chapter:introduction}
\section{Motivation}
The fast paced progress in the area of hardware and software systems that has been witnessed over the last half-century has allowed them to transcend their original limited application fields. Available processing power has steadily increased for the last decades and with it, new challenges have surfaced. Even though consistently more components have been packed into integrated circuits, their size has mostly remained the same or, in some cases, even shrunk. This has facilitated the integration of high powered processors in many applications but as a result large amounts of power dissipation can be observed, mostly in the form of heat. This represents a major challenge in embedded systems which are often very limited in their use of hardware cooling devices due to their environment. Recent developments in fields like the automotive industry have sparked a high demand for powerful CPUs to be used for applications such as autonomous driving and real-time obstacle recognition, forcing these circuits to their thermal limits. A particularly tough challenge is faced by safety critical applications since they are required to guarantee a certain quality of service under all circumstances, thus a careful balance between performance and temperature has to be found. The common temperature control approaches used in many of these systems today likely won't be able to satisfy the thermal and performance requirements of the future. In order to keep up with the progress of available processing power current approaches to temperature control have to be reexamined and new ideas evaluated.\\
\hspace*{0.5ex}\hspace{0.5ex} Numerous approaches to tackle this problem on a software level have been studied (see chapter \ref{chapter:background}) and one novel idea will be analyzed in the course of this thesis. Zhou developed GMPT \cite{Zhou2017}, a method that employs a genetic algorithm to find a static schedule which minimizes the processor peak temperature by altering the CPU frequency throughout the schedule while ensuring that all jobs still meet their deadline. To get a better understanding of the advantages and drawbacks of GMPT, a study on it has been conducted, including experiments on two hardware platforms under conditions which closely resemble real-time systems.
\section{Objectives}
The overall goal of this thesis is the analysis and evaluation of the GMPT algorithm. The individual objectives are summarized as follows:
\begin{itemize}
\item Extraction and assessment of the thermal behavior of a notebook platform. This includes adaption of the employed framework and environment to minimize potential interference which may falsify the results.
\item Extraction and assessment of the thermal behavior of a Raspberry Pi 3B. The same framework that has been used on the notebook will be employed here, however,  some code rework is required to maintain equal functionality on this new platform.
\item Execution of frequency scaling schedules which were computed by different algorithms. The results obtained by running these experiments will serve to evaluate the GMPT algorithm and compare it to a similar approach. An assessment of the schedules generated by GMPT will ultimately be provided.
\end{itemize}
\section{Outline}
The remainder of this thesis is organized in five chapters. Chapter 2 provides background information on various concepts which form the basis of this thesis. It introduces ideas which constitute the starting points for the presented work and gives an overview of related work. Chapter 3 gives insight into the environment used for the experiments, the adjustments that had to be made to it and illustrates the groundwork that was required to successfully run the experiments. Chapter 4 presents the data that was gathered from the experiments, compares the results for the different algorithms and finally assesses the performance of both approaches. Chapter 5 provides a summary of the contributions of this thesis. Finally, Chapter 6 concludes this thesis by giving an outlook into the possible future developments of GMPT and Dynamic Thermal Management as a whole.