% !TeX root = ../main.tex
% Add the above to each chapter to make compiling the PDF easier in some editors.

\chapter{Conclusion}\label{chapter:conclusion}
This thesis examined a DVFS approach to minimize the processor peak temperature in two different hardware environments.\\
\hspace*{0.5ex}\hspace{0.5ex} Initially the framework to be used for future experiments was presented. As part of the setup procedure several changes and adjustments to the framework had to be made. The need for these changes arose in part from the urge to ultimately generate accurate and authentic results which could be reproduced and in part from the necessity of porting the framework to the second hardware platform.\\
\hspace*{0.5ex}\hspace{0.5ex} With the framework in working condition on both hardware platforms, the thermal profile, i.e. the behavior of the CPU temperature for different operating frequencies was determined. This data could then be used by a set of algorithms to calculate appropriate schedules for both environments with the aim of minimizing the processor peak temperature.\\
\hspace*{0.5ex}\hspace{0.5ex} In the final part of this thesis two algorithms (GMPT and PMPT) were used to calculate aforementioned schedules for different workloads in single-task and multi-task simulations. These schedules were then employed in a series of experiments performed on both hardware platforms. The data gathered in the experiments and specifically the peak temperature achieved for different scheduling schemes was finally used to compare the performance of the GMPT approach to the PMPT method and assess the benefits and drawbacks of the GMPT algorithm.
\afterpage{\null\newpage}