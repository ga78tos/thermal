\afterpage{\blankpage}
\chapter{\abstractname}

%TODO: Abstract
The growing demand for sophisticated hardware and software systems has led to the development of high-powered processors. The large number of on-chip components results in an increasingly high power density and consequently a rising power dissipation. These high operating temperatures caused by such development have an adverse effect on reliability and longevity of the system. Especially in the area of real-time systems these factors present a major challenge which continues to interfere with progress.\\
\hspace*{0.5ex}\hspace{0.5ex} To tackle these challenges, a variety of different approaches have been studied and implemented. This thesis will examine a novel approach which aims at minimizing the CPU temperature for hard real-time systems with schedules derived by a genetic algorithm. The schedules attempt to achieve this by scaling the processor frequency and voltage throughout their execution. With the support of a scheduling framework the results given by this approach will be tested alongside other state-of-the-art thermal management methods. The experiments will be performed on two different hardware platforms, in an environment designed to simulate an actual hard real-time system. Based on temperature measurements taken during the experiments the performance of the genetic algorithm will be evaluated and the effectiveness of its schedules compared to other approaches. Ultimately an assessment of the benefits and shortcomings of this new approach will be provided.
\leavevmode\thispagestyle{empty}\newpage